\documentclass[a4paper, 12pt]{article}
\usepackage[utf8]{inputenc}
\usepackage[portuguese]{babel}

\usepackage{mathptmx}
\usepackage{enumitem}
\usepackage{amsmath,amssymb,exscale}
\usepackage{textcomp}

\usepackage{multirow}

% Margins
\usepackage{geometry}
\geometry{left=30mm,right=20mm,%
bindingoffset=0mm, top=30mm,bottom=20mm,includefoot}

% Images
\usepackage{graphicx}
\graphicspath{ {./images/} }

% Title font size
\usepackage{titlesec}
\titleformat*{\section}{\bfseries}

% Indent first sentence of a paragraph
\usepackage{indentfirst}

% Page numbers at bottom right
\usepackage{fancyhdr}
\pagestyle{fancy}
\fancyhf{}
\renewcommand{\headrulewidth}{0pt}
\rfoot{\thepage}

% 1.5 line spacing (yes, 1.25 is actually 1.5)
\fontsize{10}{12}
\linespread{1.50}
\fontfamily{ptm}
\selectfont

% Fix vspacing for section titles
\let\oldsection\section
\renewcommand{\section}[1]{{\vspace{20pt}\oldsection*{#1}\vspace{-10pt}}}

% Indent size
\setlength\parindent{10mm}

% Jump 20pt
\newcommand{\jump}[1]{{\vspace{20pt}}}

\newenvironment{leftquote}[1][]
{
    \hspace{30pt} #1
    \vspace{20pt}
    \fontsize{10}{10}\selectfont
    \begin{flushright}
    \begin{minipage}{120mm}
}
{
    \end{minipage}
    \end{flushright}
    \par
    \vspace{20pt}
}

\newcommand{\tmp}{}

\newenvironment{container}[3]
{
    \renewcommand{\tmp}{#3} %gambiarra
    
    \begin{center}
    \begin{minipage}{#1}
    \centering
    #2
}
{
    \raggedright
    {\footnotesize \tmp}
    \end{minipage}
    \end{center}
}


\begin{document}

% Remove page number
\clearpage
\thispagestyle{empty}

\begin{bfseries}
\begin{center}

\includegraphics[scale=0.45]{ufc.png} \\
\vspace{-4pt} 
UNIVERSIDADE FEDERAL DO CEARÁ \\
\vspace{4pt} 
CENTRO DE TECNOLOGIA \\
\vspace{4pt} 
DEPARTAMENTO DE TELEINFORMÁTICA \\
\vspace{4pt}
a
\vspace{4pt}
SEMESTRE 2025.1 \\


\vspace*{\fill}
\textbf{Relatório dos Homeworks de Álgebra Linear e Multilinear}
\vspace*{\fill}

\end{center}

\begin{itemize}[leftmargin=*]
    \setlength{\itemsep}{0pt}
    \item[] ALUNO: Ruan Pereira Alves
    \item[] MATRÍCULA: 569551
\end{itemize}

\end{bfseries}
\newpage

\section{Homework 01}

Em álgebra linear, podemos verificar como podemos fazer a resolução de sistemas lineares de forma um pouco mais dinâmica, e especialmente como é possível realizar operações utilizando vetores coluna, o que permite interpretar as operações utilizando as colunas do sistema, mantendo o formato $Ax = b$. 

Um espaço vetorial é constituído de vetores que possuem as operações de adição de vetores e multiplicação de escalares, e que alguns axiomas devem ser seguidos para que as operações ocorram. O maior exemplo de um espaço vetorial é o espaço $R^n$, que pode possuir $n$ dimensões.

Um subespaço seria apenas um subconjunto de vetores dentro de um espaço vetorial, vide exemplo $R^3$.

Após isso, precisamos verificar se um dado conjunto de vetores possui independência linear, ou seja se os vetores só possuem uma forma de serem zerados, sendo "forçados" a serem zero. Em outras palavras, o espaço nulo da matriz A com os vetores só possui o vetor zero.

Com essa independência, podemos assim gerar um espaço com base no conjunto de vetores, em que as combinações destes vetores criam o espaço vetorial, ou seja baseado em operações entre esses vetores. O espaço vai consistir de todas as possíveis combinações lineares do conjunto de vetores. 

Uma base seria uma sequência de vetores que justamente possui as duas propriedades mencionadas acima. Para qualquer espaço, o número de vetores-base é uma propriedade do próprio espaço, ou seja para qualquer número de vetores-base de um espaço, o mesmo número de vetores é contido dentro do espaço, o que nos traz o conceito de dimensão. 

Para a verificação de um vetor arbitrário em um subespaço definido, é necessário realizar as operações vetoriais com os vetores-base do subespaço, buscando montar o vetor arbitrário a partir dos vetores-base, ou seja criando um sistema linear e o resolvendo. Se houver solução, o vetor pertence. 

Assim, foi possível realizar o desenvolvimento da atividade, vide o código a seguir: 

\begin{lstlisting}
	#include <iostream>
	#include <armadillo>
	
	using namespace std;
	using namespace arma;
	
	int main () {
		mat A = {{1,2,3},
			{4,5,6},
			{7,8,9}};
		
		cout << "hello" << A << endl;
		double det_A = det(A);
		cout << "det A = " << det_A << endl;
		mat B = {{3,2,1}, {0,0,0}, {0,0,0}};
		cout << A + B << endl;
		
		mat C = A * 2;
		cout << C << endl; 
		uword r = arma::rank(A);
		cout << r << endl;
		
		mat base = {{1,0,0}, {0,1,0},{0,0,1}};
		vector<double> arb = {2,2,1};
		
		return 0;
	}
	
\end{lstlisting}

\section{Homework 02 linear}

\section{Homework 03 linear}

\section{Homework 09 multilinear}

A ideia é decompor um tensor de ordem superior (neste caso, um tensor de terceira ordem $\mathcal{X}$) em uma soma de $R$ componentes de rank-1. Cada componente de rank-1 é o produto externo de $R$ vetores, ou seja o rank do tensor.

Onde $\mathbf{a}_r$, $\mathbf{b}_r$ e $\mathbf{c}_r$ são os $r$-ésimos vetores das matrizes fatoriais A, B e C, respectivamente, e $\circ$ denota o produto externo. Na prática, isso é equivalente a encontrar as matrizes fatoriais A, B e C.

\end{document}
