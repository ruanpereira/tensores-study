\documentclass[a4paper, 12pt]{article}
\usepackage[utf8]{inputenc}
\usepackage[portuguese]{babel}

\usepackage{mathptmx}
\usepackage{enumitem}
\usepackage{amsmath,amssymb,exscale}
\usepackage{textcomp}

\usepackage{multirow}

% Margins
\usepackage{geometry}
\geometry{left=30mm,right=20mm,%
bindingoffset=0mm, top=30mm,bottom=20mm,includefoot}

% Images
\usepackage{graphicx}
\graphicspath{ {./images/} }

% Title font size
\usepackage{titlesec}
\titleformat*{\section}{\bfseries}

% Indent first sentence of a paragraph
\usepackage{indentfirst}

% Page numbers at bottom right
\usepackage{fancyhdr}
\pagestyle{fancy}
\fancyhf{}
\renewcommand{\headrulewidth}{0pt}
\rfoot{\thepage}

% 1.5 line spacing (yes, 1.25 is actually 1.5)
\fontsize{10}{12}
\linespread{1.50}
\fontfamily{ptm}
\selectfont

% Fix vspacing for section titles
\let\oldsection\section
\renewcommand{\section}[1]{{\vspace{20pt}\oldsection*{#1}\vspace{-10pt}}}

% Indent size
\setlength\parindent{10mm}

% Jump 20pt
\newcommand{\jump}[1]{{\vspace{20pt}}}

\newenvironment{leftquote}[1][]
{
    \hspace{30pt} #1
    \vspace{20pt}
    \fontsize{10}{10}\selectfont
    \begin{flushright}
    \begin{minipage}{120mm}
}
{
    \end{minipage}
    \end{flushright}
    \par
    \vspace{20pt}
}

\newcommand{\tmp}{}

\newenvironment{container}[3]
{
    \renewcommand{\tmp}{#3} %gambiarra
    
    \begin{center}
    \begin{minipage}{#1}
    \centering
    #2
}
{
    \raggedright
    {\footnotesize \tmp}
    \end{minipage}
    \end{center}
}


\usepackage{listings}
\usepackage{amsmath,amssymb,caption}
%% my poor man's solution to arc notation
\newcommand{\tarc}{\mbox{\large$\frown$}}
\newcommand{\arc}[1]{\stackrel{\tarc}{#1}}

\lstset{frame=tb,
  language=Matlab,
  aboveskip=3mm,
  belowskip=3mm,
  showstringspaces=false,
  columns=flexible,
  basicstyle={\small\ttfamily},
  numbers=none,
  numberstyle=\tiny\color{gray},
  keywordstyle=\color{blue},
  commentstyle=\color{dkgreen},
  stringstyle=\color{mauve},
  breaklines=true,
  breakatwhitespace=true,
  tabsize=3
}

\begin{document}

% Remove page number
\clearpage
\thispagestyle{empty}

\begin{bfseries}
\begin{center}

\includegraphics{ufc.png} \\
\vspace{-4pt} 
UNIVERSIDADE FEDERAL DO CEARÁ \\
\vspace{4pt} 
CENTRO DE TECNOLOGIA \\
\vspace{4pt} 
DEPARTAMENTO DE ENGENHARIA DE TELEINFORMÁTICA \\
\vspace{4pt}
SINAIS E SISTEMAS \\
\vspace{4pt}
SEMESTRE 2023.2 \\


\vspace*{\fill}
\textbf{Lista de Exercícios Módulo 01 (parte 2)}
\vspace*{\fill}

\end{center}

\begin{itemize}[leftmargin=*]
    \setlength{\itemsep}{0pt}
    \item[] ALUNO: Ruan Pereira Alves
    \item[] MATRÍCULA: 510626
    \item[] PROFESSOR: IGOR MOACO GUERREIRO
    \item[] DISCIPLINA: SINAIS E SISTEMAS
    \item[] DATA DE ENTREGA DO RELATÓRIO: 02/10/2023
\end{itemize}

\end{bfseries}

\newpage

\section{QUESTIONÁRIO}

\textbf{Questão 01} 

a) $\alpha \neq \beta$: \\

Temos que 
\begin{equation}
        y[n] = \sum ^{+\infty} _{k = -\infty} x[k]h[n - k] 
\end{equation}
Como $u[k]$ representa o degrau unitário, e temos que:

\begin{equation}
    \begin{cases}
     u[n] = 0, n < 0 \\
     u[n] = 1, n \geq 0 
    \end{cases} 
\end{equation}

ficamos no fim com

\begin{equation}\label{eq:ref1}
    y[n] = \sum ^{+\infty} _{k = 0} \alpha^k \beta^{n-k}.
\end{equation}

Isolando as variáveis dependentes de $k$ na expressão, desta forma, 
$\beta^n$ será constante dentro do somatório, podemos retirar esta parcela do 
somatório, portanto, obtemos:

\begin{equation}
    y[n] = \beta^n \sum _{k = 0} ^{n} (\alpha/\beta)^k,
\end{equation}

Mas temos que 

\begin{equation}
    \sum ^{n} _{k=0} \alpha ^k = \frac{1 - \alpha^{n-1}}{1 - \alpha}
\end{equation}

Substituindo $\alpha^k$ por $(\alpha/\beta)^k$, teremos no final:

\begin{equation}
    y[n] = \left( \frac{\beta^{n+1} - \alpha^{n+1}}{\beta - \alpha} \right)u[n].
\end{equation}

b) $\alpha = \beta$:\\

A partir do item anterior, com o mesmo desenvolvimento até a equação 
\ref{eq:ref1}. Utilizando que $\alpha = \beta$, temos:

\begin{equation}
    y[n] = \sum_{k = 0} ^n \alpha ^n u[k] 
\end{equation}

Assim:

\begin{equation}
    y[n] = (n + 1)\alpha^n u[n].
\end{equation}

\textbf{Questão 02}

Temos que a convolução é:

\begin{equation}
    y[n] = \sum ^{+\infty} _{k = -\infty} x[k]h[n - k]
\end{equation}

Assim, aplicando em $x(t)$ e $h(t)$, temos:
\begin{equation}
    y(t) = \int _{-\infty} ^{\infty} u(\tau) - 2u(\tau - 3) + u(\tau - 6) e^{(2(t - \tau))}u(1- (t - \tau)) d\tau,
\end{equation}

Mas

\begin{equation}
    e^{(2(t - \tau))}u(1 - (t - \tau)) = \begin{cases}
        e^{(2(t- \tau))}, \text{se } \tau > t-1 , \\
        0 \;, \text{caso contrário.}
    \end{cases}
\end{equation}
Dividiremos nossa expressão para $y(t)$ em uma soma de integrais:

\begin{equation}
    y(t) = \int _{0} ^3 e^{(2(t - \tau))} d\tau - \int_{3} ^6 e^{(2(t - \tau))}d\tau,
\end{equation}

desta forma

\begin{equation}
    y(t) = \left[ -1/2 (e^{2t} - e^{2\tau})\right]|_{\tau = 0} ^3 - \left[ -1/2 (e^{2t} - e^{2\tau})\right]|_{\tau = 3} ^6, 
\end{equation}

Portanto,

\begin{equation}
    y(t) = \frac{1}{2}(-2 e^{2(t-3)} + e^{2t} + e^{2(t-6)}),
\end{equation}



\newpage

\textbf{Questão 03} 

Com o auxílio de ferramentas de IA (Inteligência Artificial), criamos um questionário de 20 perguntas que buscam reunir:

\begin{itemize}
    \item Dados de idade, engenharia cursada e semestre
    \item Conhecimento acerca do funcionamento básico do CREA
    \item Entender qual é o nível de divulgamento do CREA dentro do DETI
    \item Saber qual a opinião dos alunos acerca das funções do CREA e do divulgamento no departamento
\end{itemize}

Essas informações são importantes e foram escolhidas para serem coletadas pela pesquisa pois nos ajudam a determinar qual é o contingente que pessoas que estamos entrevistando, se é divulgado o CREA pelo DETI-UFC, se há algum nível de difusão de informação, e por fim, se os alunos estão interessados no CREA.


\newpage

\textbf{Questão 04} 

A partir de

\begin{equation}
    y(t) = \int_\infty ^t \exp{(-(t-\tau))} \;x(\tau - 2)d\tau
\end{equation}

a) 

Analisando os limites de integração e a equação, podemos reescrever:

\begin{equation}
    h(t) = \exp{(-(t-2))}u(t-2)
\end{equation}

Para a integração entre $-\infty$ a $t$, precisamos  que a parte exponencial de $h(t)$ seja multiplicada por um degrau unitário, que esteja no 
mesmo domínio de tempo que o sinal $x(t)$ está.\\



\newpage
\textbf{Questão 05}

Temos que 

\begin{equation}
    \frac{dx(t)}{dt} = -6e^ {-3t}u(t-1) + 2\delta(t-1)
\end{equation}


simplificando, teremos
$-3x(t) + 2\delta(t-1)$.\\

Desenvolvendo:
\begin{equation}
    -3y(t) + e^{-2t}u(t) = (-3x(t) + 2\delta(t-1))*h(t) \therefore \\ 2h(t-1) = e^{-2t}u(t)
\end{equation}



que será o resultado. 

Reorganizando a equação acima, temos:

\begin{equation}
    h(t) = \frac{1}{2}e^{-2(t+1)}u(t+1).
\end{equation}

\newpage
\textbf{Questão 06}


\end{document}
